\documentclass{article}

\usepackage{polski} % Pozwala na użycie polskiego. Ustawia między innymi fontenc na T1
\usepackage[utf8]{inputenc} % Informuje o kodowaniu

\usepackage{graphicx}
\graphicspath{ {./Obrazy/} }

\usepackage{textcomp} % Znaki specjalne takie jak ~

\title{Laboratorium sieci komputerowych -- C1 \\ Wirtualizacja i zdalny dostęp}
\author{Krzysztof Dąbrowski gr. 3}
\date{\today}

\begin{document}
\maketitle{}
\tableofcontents{}
\newpage

\section{Wstęp}
Laboratorium \texttt{c1} było podzielone na dwa spotkania. Celem pierwszego było zapoznanie się z mechanizmem tworzenia i korzystania z maszyn wirtualnych. Podczas drugiego spotkania możliwe było przećwiczenie zdalnego korzystania z interfejsów graficznych.

\section{Wirtualizacja}
Technologia wirtualizacji umożliwia by komputer symulował działanie wybranej maszyny tak jakby była fizycznym komputerem. System przeprowadzający symulację nazywany jest \textit{gospodarzem}, natomiast system symulowany nazywany jest \textit{gościem}.

Do tworzenia i zarządzania wirtualnymi maszynami gospodarz potrzebuje specjalnego programu -- wirtualtizatora.

\paragraph{Powszechnie stosowane wirtualizatory:}
\begin{itemize}
    \item KVM
    \item Virtual box
    \item Bhyve
    \item Hyper-V
    \item VMWare
\end{itemize}

\subsection{Instalacja wirtualizatora}



\end{document}